\documentclass{article}

\usepackage{tabularx}
\usepackage{booktabs}

\title{Problem Statement and Goals\\\progname}

\author{\authname}

\date{}

%% Comments

\usepackage{color}

\newif\ifcomments\commentstrue %displays comments
%\newif\ifcomments\commentsfalse %so that comments do not display

\ifcomments
\newcommand{\authornote}[3]{\textcolor{#1}{[#3 ---#2]}}
\newcommand{\todo}[1]{\textcolor{red}{[TODO: #1]}}
\else
\newcommand{\authornote}[3]{}
\newcommand{\todo}[1]{}
\fi

\newcommand{\wss}[1]{\authornote{magenta}{SS}{#1}} 
\newcommand{\plt}[1]{\authornote{cyan}{TPLT}{#1}} %For explanation of the template
\newcommand{\an}[1]{\authornote{cyan}{Author}{#1}}

%% Common Parts

\newcommand{\progname}{ProgName} % PUT YOUR PROGRAM NAME HERE
\newcommand{\authname}{Team \#, Team Name
\\ Student 1 name
\\ Student 2 name
\\ Student 3 name
\\ Student 4 name} % AUTHOR NAMES                  

\usepackage{hyperref}
    \hypersetup{colorlinks=true, linkcolor=blue, citecolor=blue, filecolor=blue,
                urlcolor=blue, unicode=false}
    \urlstyle{same}
                                


\begin{document}

\maketitle

\begin{table}[hp]
\caption{Revision History} \label{TblRevisionHistory}
\begin{tabularx}{\textwidth}{llX}
\toprule
\textbf{Date} & \textbf{Developer(s)} & \textbf{Change}\\
\midrule
Date1 & Name(s) & Description of changes\\
Date2 & Name(s) & Description of changes\\
... & ... & ...\\
\bottomrule
\end{tabularx}
\end{table}

\section{Problem Statement}

\wss{You should check your problem statement with the
\href{https://github.com/smiths/capTemplate/blob/main/docs/Checklists/ProbState-Checklist.pdf}
{problem statement checklist}.} 

\wss{You can change the section headings, as long as you include the required
information.}

\subsection{Problem}

\subsection{Inputs and Outputs}

\wss{Characterize the problem in terms of ``high level'' inputs and outputs.  
Use abstraction so that you can avoid details.}

\subsection{Stakeholders}

\subsection{Environment}

\wss{Hardware and Software Environment}

\section{Goals}

\section{Stretch Goals}

\section{Extras}

\wss{For CAS 741: State whether the project is a research project. This
designation, with the approval (or request) of the instructor, can be modified
over the course of the term.}

\wss{For SE Capstone: List your extras.  Potential extras include usability
testing, code walkthroughs, user documentation, formal proof, GenderMag
personas, Design Thinking, etc.  (The full list is on the course outline and in
Lecture 02.) Normally the number of extras will be two.  Approval of the extras
will be part of the discussion with the instructor for approving the project.
The extras, with the approval (or request) of the instructor, can be modified
over the course of the term.}

\newpage{}

\section*{Appendix --- Reflection}

\wss{Not required for CAS 741}

The purpose of reflection questions is to give you a chance to assess your own
learning and that of your group as a whole, and to find ways to improve in the
future. Reflection is an important part of the learning process.  Reflection is
also an essential component of a successful software development process.  

Reflections are most interesting and useful when they're honest, even if the
stories they tell are imperfect. You will be marked based on your depth of
thought and analysis, and not based on the content of the reflections
themselves. Thus, for full marks we encourage you to answer openly and honestly
and to avoid simply writing ``what you think the evaluator wants to hear.''

Please answer the following questions.  Some questions can be answered on the
team level, but where appropriate, each team member should write their own
response:


\begin{enumerate}
    \item What went well while writing this deliverable? 
    \item What pain points did you experience during this deliverable, and how
    did you resolve them?
    \item How did you and your team adjust the scope of your goals to ensure
    they are suitable for a Capstone project (not overly ambitious but also of
    appropriate complexity for a senior design project)?
\end{enumerate}  



\subsubsection*{Omar}
\hspace{2em} What went well the most would be that everyone was on the same page for this project and so coming up with what we wanted to accomplish was pretty easy and we didn't really have many disagreements. This made it easy to focus on the task of writing the problem statement and goals rather than having to worry about different opinions on the team. This project was also one of the projects that was on the list of projects pdf given to us so it was easy to figure our the problem statement and goals since we didn't really have much to come up with anything on our own. 

\hspace{2em} One challenge was avoiding either too much technical detail or too much abstraction. We really had to focus on what we wanted to convey and add too much detail as to not contrain our selves but also not have too little detail that it isn't clear as to what we are doing. We initially struggled to find the right balance between scholarly needs (manuscript context) and technical specifications (ML algorithms, environment). We resolved this by starting broad, then refining with feedback and checking against the POC and problem statement checklists. That way we made sure that our POC and problem statement were in line with each other. Another pain point was uncertainty about which machine learning techniques would realistically be feasible; to address this, we distinguished between core goals and stretch goals to avoid overcommitting. This way we also didn't contrain oursleves later down the line when we start implementing the solution.

\hspace{2em} Initially, we considered a full end-to-end system covering Sanskrit, Tibetan, and Chinese manuscripts. We recognized this was quite ambitious especially given the time and resources that we had. Instead, we narrowed our core scope to Sanskrit fragments only, focusing on orientation correction, edge/damage-based matching, and preliminary script identification. Transcription and cross-lingual extensions were moved into stretch goals as they are not the core goals that we are trying to achieve with this project. If we are ahead of schedule then they would be great additions to add to the project but again as mentioned they are not the core focus of this project. This adjustment ensures the project is challenging enough, but achievable within the Capstone timeline.


\vspace{0.5cm}

\subsubsection*{Yousef}

\vspace{0.5cm}

\subsubsection*{Aswin}

\vspace{0.5cm}

\subsubsection*{Dylan}

\vspace{0.5cm}

\subsubsection*{Umar}



\end{document}