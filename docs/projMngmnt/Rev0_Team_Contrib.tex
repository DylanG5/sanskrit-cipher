\documentclass{article}

\usepackage{float}
\restylefloat{table}

\usepackage{booktabs}

\title{Team Contributions: Rev 0\\\progname}

\author{\authname}

\date{}

\input{../Comments}
%% Common Parts

\newcommand{\progname}{Software Engineering} % PUT YOUR PROGRAM NAME HERE
\newcommand{\authname}{Team \#1, Sanskrit Ciphers
\\ Omar El Aref
\\ Dylan Garner
\\ Muhammad Umar Khan
\\ Aswin Kuganesan
\\ Yousef Shahin} % AUTHOR NAMES                  

\usepackage{hyperref}
    \hypersetup{colorlinks=true, linkcolor=blue, citecolor=blue, filecolor=blue,
                urlcolor=blue, unicode=false}
    \urlstyle{same}
                                


\begin{document}

\maketitle

This document summarizes the contributions of each team member for the Rev 0
Demo.  The time period of interest is the time between the PoC demo and the Rev
0 demo; the contributions prior to the PoC are NOT included.

\section{Demo Plans}

In this demo, we will showcase the current functionality implemented across both the user interface and machine learning components of the system. This includes recent UI updates such as improved search and filter functionality, dynamic scaling and interaction on the canvas, and fragment annotation support. We will also demonstrate the end-to-end ML pipeline, highlighting fragment segmentation, line count detection, circle classification, script type identification, and scaling behavior. Together, these demos illustrate how the UI and ML components integrate to support efficient fragment exploration and analysis.

\section{Team Meeting Attendance}

\begin{table}[H]
\centering
\begin{tabular}{ll}
\toprule
\textbf{Student} & \textbf{Meetings}\\
\midrule
Total & 3\\
Omar El Aref & 3\\
Dylan Garner & 3\\
Muhammad Umar Khan & 3\\
Aswin Kuganesan & 3\\
Yousef Shahin & 3\\
\bottomrule
\end{tabular}
\end{table}

\section{Supervisor/Stakeholder Meeting Attendance}

\noindent \textbf{Supervisor's Name: } Dr. Shayne Clarke 

\begin{table}[H]
\centering
\begin{tabular}{ll}
\toprule
\textbf{Student} & \textbf{Meetings}\\
\midrule
Total & 2\\
Omar El Aref & 2\\
Dylan Garner & 2\\
Muhammad Umar Khan & 2\\
Aswin Kuganesan & 2\\
Yousef Shahin & 2\\
\bottomrule
\end{tabular}
\end{table}

\section{Lecture Attendance}

\begin{table}[H]
\centering
\begin{tabular}{ll}
\toprule
\textbf{Student} & \textbf{Lectures}\\
\midrule
Total & Num\\
Omar ElAref & 1\\
Muhammad Umar Khan & 0\\
Dylan Garner & 0\\
Aswin Kuganesan  & 0\\
Yousef Shahin & 0\\
\bottomrule
\end{tabular}
\end{table}


\section{TA Document Discussion Attendance}

\noindent \textbf{TA's Name: } Tanya Djavaherpour

\begin{table}[H]
\centering
\begin{tabular}{ll}
\toprule
\textbf{Student} & \textbf{Lectures}\\
\midrule
Total & 3\\
Omar El Aref & 3\\
Dylan Garner & 3\\
Muhammad Umar Khan & 3\\
Aswin Kuganesan & 2\\
Yousef Shahin & 2\\
\bottomrule
\end{tabular}
\end{table}

Some members of the team missed the last TA document discussions due to having a midterm that was scheduled right after that discussion. However, the team members that attended the discussion talked about the team's concerns with the TA and shared the discussion notes with the rest of the team after.

\section{Commits}


\begin{table}[H]
\centering
\begin{tabular}{lll}
\toprule
\textbf{Student} & \textbf{Commits} & \textbf{Percent}\\
\midrule
Total & 335 & 100\% \\
Omar El Aref & 11 & 30.6\% \\
Dylan Garner & 19 & 52.7\% \\
Muhammad Umar Khan & 3 & 8.3\% \\
Aswin Kuganesan & 1 & 2.8\% \\
Yousef Shahin & 2 & 5.6\% \\
\bottomrule
\end{tabular}
\end{table}

The difference in the number of commits between team members can be attributed to the varying size of each commit, as some members may make larger, more comprehensive commits while others commit more frequently with smaller changes. Additionally, some team members have ongoing work that has not yet been merged into the main repository, which is not reflected in these commit counts.

\section{Issue Tracker}

\begin{table}[H]
\centering
\begin{tabular}{lll}
\toprule
\textbf{Student} & \textbf{Authored (O+C)} & \textbf{Assigned (C only)}\\
\midrule
Omar El Aref & 21 & 14 \\
Dylan Garner & 8 & 5 \\
Muhammad Umar Khan & 22 & 9 \\
Aswin Kuganesan & 5 & 6 \\
Yousef Shahin & 4 & 7 \\
\bottomrule
\end{tabular}
\end{table}

The variation in authored issues is primarily due to differences in issue creation practices, with some team members creating more granular issues for tracking meetings and smaller tasks. The assigned and completed issues show more balanced distribution across team members, reflecting consistent contribution to closing issues during the Rev 0 period.

\section{CICD}

Our project utilizes GitHub Actions for Continuous Integration and Continuous Deployment (CI/CD) in the following ways:

\begin{enumerate}
\item \textbf{Automated LaTeX Compilation}: A CI/CD workflow automatically compiles LaTeX documentation files whenever changes are pushed to the \texttt{docs/} directory. The workflow:
\begin{itemize}
\item Uses TeXLive 2024 with full scheme
\item Detects changed .tex files and compiles only those files
\item Automatically commits generated PDFs back to the repository
\item Runs on both push and pull request events to the main branch
\end{itemize}

\item \textbf{Database Repopulation}: Automated workflows will repopulate sample databases after changes are made to the underlying models and logic that generates metadata for each manuscript fragment image. This ensures database schemas and sample data remain consistent with code changes.

\item \textbf{Linting \& Automated Testing}: We will automate linting according to the appropriate style guides, as well as execute our automated test suites where applicable on a PR.
\end{enumerate}

\section{Team Charter Trigger Items}

The team charter had the following quantified triggers:
\begin{itemize}
  \item \textbf{Attendance Triggers:} All team members were expected to attend all meetings unless a valid excuse was provided in advance. 
  This included team meetings and supervisor meetings, and missing a meeting without a valid excuse would count as a violation. This 
  also includes leaving a meeting early or being late without a valid excuse.
  \item \textbf{Performance Triggers:} Team members were expected to complete their assigned tasks on time and participate in providing code reviews and feedback. 
  All team members were expected to equally contribute to the project, and the team was expected to work together to ensure that the team's performance met the project goals.
  \item \textbf{Communication Triggers:} Team members were expected to equally participate in group discussions, communicate effectively with each other, and respect each other's opinions.
  Team members were also expected to follow the code of conduct outlined in the team charter, which included treating each other with respect and professionalism. 
  \item \textbf{Decision-Making Triggers:} Team members were expected to actively participate in decision-making processes, which involved using a consensus-based approach for decision-making. 
  No team member was to take control of the decision-making process in any circumstance.
\end{itemize} 

There have been the following violations of the triggers:
\begin{itemize}
  \item \textbf{Decision-Making:} A few limited-scope decisions were made by a subset of the team without full consensus. These did not affect major milestones but did not align with the charter process.
  \item \textbf{Performance:} Some tasks were not completed as early as expected, reducing buffer time before milestones. While all milestones were still met, the team finished with less time available than planned.
\end{itemize}

To address these violations, the team plans to:
\begin{itemize}
  \item \textbf{Adjust Decision-Making Triggers:} For decisions that do not require full-team input, a smaller subset may proceed. Consensus-based decision-making will still be used for major decisions that affect the entire team.
  \item \textbf{Prevent Violation of Performance Triggers:} We will reinforce adherence to the project timeline and ensure that all team members understand their responsibilities and deadlines.
\end{itemize}

\section{Additional Productivity Metrics}

\wss{If your team has additional metrics of productivity, please feel free to
add them to this report.}

\end{document}