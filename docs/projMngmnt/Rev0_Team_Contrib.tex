\documentclass{article}

\usepackage{float}
\restylefloat{table}

\usepackage{booktabs}

\title{Team Contributions: Rev 0\\\progname}

\author{\authname}

\date{}

\input{../Comments}
%% Common Parts

\newcommand{\progname}{Software Engineering} % PUT YOUR PROGRAM NAME HERE
\newcommand{\authname}{Team \#1, Sanskrit Ciphers
\\ Omar El Aref
\\ Dylan Garner
\\ Muhammad Umar Khan
\\ Aswin Kuganesan
\\ Yousef Shahin} % AUTHOR NAMES                  

\usepackage{hyperref}
    \hypersetup{colorlinks=true, linkcolor=blue, citecolor=blue, filecolor=blue,
                urlcolor=blue, unicode=false}
    \urlstyle{same}
                                


\begin{document}

\maketitle

This document summarizes the contributions of each team member for the Rev 0
Demo.  The time period of interest is the time between the PoC demo and the Rev
0 demo; the contributions prior to the PoC are NOT included.

\section{Demo Plans}

\wss{What will you be demonstrating}

\section{Team Meeting Attendance}

\wss{For each team member how many team meetings have they attended over the
time period of interest.  This number should be determined from the meeting
issues in the team's repo.  The first entry in the table should be the total
number of team meetings held by the team.}

\begin{table}[H]
\centering
\begin{tabular}{ll}
\toprule
\textbf{Student} & \textbf{Meetings}\\
\midrule
Total & Num\\
Name 1 & Num\\
Name 2 & Num\\
Name 3 & Num\\
Name 4 & Num\\
Name 5 & Num\\
\bottomrule
\end{tabular}
\end{table}

\wss{If needed, an explanation for the counts can be provided here.}

\section{Supervisor/Stakeholder Meeting Attendance}

\wss{For each team member how many supervisor/stakeholder team meetings have
they attended over the time period of interest.  This number should be determined
from the supervisor meeting issues in the team's repo.  The first entry in the
table should be the total number of supervisor and team meetings held by the
team.  If there is no supervisor, there will usually be meetings with
stakeholders (potential users) that can serve a similar purpose.}

\noindent \textbf{Supervisor's Name: } [fill in this information]

\begin{table}[H]
\centering
\begin{tabular}{ll}
\toprule
\textbf{Student} & \textbf{Meetings}\\
\midrule
Total & Num\\
Name 1 & Num\\
Name 2 & Num\\
Name 3 & Num\\
Name 4 & Num\\
Name 5 & Num\\
\bottomrule
\end{tabular}
\end{table}

\wss{If needed, an explanation for the counts can be provided here.}

\section{Lecture Attendance}

\wss{For each team member how many lectures have they attended over the time
period of interest.  This number should be determined from the lecture issues in
the team's repo. You can find the number of lectures in the time period of
interest by looking at the
\href{https://calendar.google.com/calendar/u/0/embed?src=rnboqiaki1k2la7rpu3bn0um58@group.calendar.google.com&ctz=America/Toronto}
{Google calendar} for the capstone course.}

\wss{NOTE: There will be approximately 1 lecture between the POC and Rev0 demos}

\begin{table}[H]
\centering
\begin{tabular}{ll}
\toprule
\textbf{Student} & \textbf{Lectures}\\
\midrule
Total & Num\\
Name 1 & Num\\
Name 2 & Num\\
Name 3 & Num\\
Name 4 & Num\\
Name 5 & Num\\
\bottomrule
\end{tabular}
\end{table}

\wss{If needed, an explanation for the lecture attendance can be provided here.}

\section{TA Document Discussion Attendance}

\wss{For each team member how many of the informal document discussion meetings
with the TA were attended over the time period of interest.}

\noindent \textbf{TA's Name: } [fill in this information]

\begin{table}[H]
\centering
\begin{tabular}{ll}
\toprule
\textbf{Student} & \textbf{Lectures}\\
\midrule
Total & Num\\
Name 1 & Num\\
Name 2 & Num\\
Name 3 & Num\\
Name 4 & Num\\
Name 5 & Num\\
\bottomrule
\end{tabular}
\end{table}

\wss{If needed, an explanation for the attendance can be provided here.}

\section{Commits}

\wss{For each team member how many commits to the main branch have been made
over the time period of interest.  The total is the total number of commits for
the entire team since the beginning of the term.  The percentage is the
percentage of the total commits made by each team member.}

\begin{table}[H]
\centering
\begin{tabular}{lll}
\toprule
\textbf{Student} & \textbf{Commits} & \textbf{Percent}\\
\midrule
Total & Num & 100\% \\
Name 1 & Num & \% \\
Name 2 & Num & \% \\
Name 3 & Num & \% \\
Name 4 & Num & \% \\
Name 5 & Num & \% \\
\bottomrule
\end{tabular}
\end{table}

\wss{If needed, an explanation for the counts can be provided here.  For
instance, if a team member has more commits to unmerged branches, these numbers
can be provided here.  If multiple people contribute to a commit, git allows for
multi-author commits.}

\section{Issue Tracker}

\wss{For each team member how many issues have they authored (including open and
closed issues (O+C)) and how many have they been assigned (only counting closed
issues (C only)) over the time period of interest.}

\begin{table}[H]
\centering
\begin{tabular}{lll}
\toprule
\textbf{Student} & \textbf{Authored (O+C)} & \textbf{Assigned (C only)}\\
\midrule
Name 1 & Num & Num \\
Name 2 & Num & Num \\
Name 3 & Num & Num \\
Name 4 & Num & Num \\
Name 5 & Num & Num \\
\bottomrule
\end{tabular}
\end{table}

\wss{If needed, an explanation for the counts can be provided here.}

\section{CICD}

\wss{Say how CICD is used in your project}

\section{Team Charter Trigger Items}

The team charter had the following quantified triggers:
\begin{itemize}
  \item \textbf{Attendance Triggers:} All team members were expected to attend all meetings unless a valid excuse was provided in advance. 
  This included team meetings and supervisor meetings, and missing a meeting without a valid excuse would count as a violation. This 
  also includes leaving a meeting early or being late without a valid excuse.
  \item \textbf{Performance Triggers:} Team members were expected to complete their assigned tasks on time and participate in providing code reviews and feedback. 
  All team members were expected to equally contribute to the project, and the team was expected to work together to ensure that the team's performance met the project goals.
  \item \textbf{Communication Triggers:} Team members were expected to equally participate in group discussions, communicate effectively with each other, and respect each other's opinions.
  Team members were also expected to follow the code of conduct outlined in the team charter, which included treating each other with respect and professionalism. 
  \item \textbf{Decision-Making Triggers:} Team members were expected to actively participate in decision-making processes, which involved using a consensus-based approach for decision-making. 
  No team member was to take control of the decision-making process in any circumstance.
\end{itemize} 

There have been the following violations of the triggers:
\begin{itemize}
  \item \textbf{Decision-Making:} A few limited-scope decisions were made by a subset of the team without full consensus. These did not affect major milestones but did not align with the charter process.
  \item \textbf{Performance:} Some tasks were not completed as early as expected, reducing buffer time before milestones. While all milestones were still met, the team finished with less time available than planned.
\end{itemize}

To address these violations, the team plans to:
\begin{itemize}
  \item \textbf{Adjust Decision-Making Triggers:} For decisions that do not require full-team input, a smaller subset may proceed. Consensus-based decision-making will still be used for major decisions that affect the entire team.
  \item \textbf{Prevent Violation of Performance Triggers:} We will reinforce adherence to the project timeline and ensure that all team members understand their responsibilities and deadlines.
\end{itemize}

\section{Additional Productivity Metrics}

\wss{If your team has additional metrics of productivity, please feel free to
add them to this report.}

\end{document}