\documentclass[12pt, titlepage]{article}

\usepackage{booktabs}
\usepackage{tabularx}
\usepackage{hyperref}
\hypersetup{
    colorlinks,
    citecolor=blue,
    filecolor=black,
    linkcolor=red,
    urlcolor=blue
}
\usepackage[round]{natbib}

%% Comments

\usepackage{color}

\newif\ifcomments\commentstrue %displays comments
%\newif\ifcomments\commentsfalse %so that comments do not display

\ifcomments
\newcommand{\authornote}[3]{\textcolor{#1}{[#3 ---#2]}}
\newcommand{\todo}[1]{\textcolor{red}{[TODO: #1]}}
\else
\newcommand{\authornote}[3]{}
\newcommand{\todo}[1]{}
\fi

\newcommand{\wss}[1]{\authornote{magenta}{SS}{#1}} 
\newcommand{\plt}[1]{\authornote{cyan}{TPLT}{#1}} %For explanation of the template
\newcommand{\an}[1]{\authornote{cyan}{Author}{#1}}

%% Common Parts

\newcommand{\progname}{ProgName} % PUT YOUR PROGRAM NAME HERE
\newcommand{\authname}{Team \#, Team Name
\\ Student 1 name
\\ Student 2 name
\\ Student 3 name
\\ Student 4 name} % AUTHOR NAMES                  

\usepackage{hyperref}
    \hypersetup{colorlinks=true, linkcolor=blue, citecolor=blue, filecolor=blue,
                urlcolor=blue, unicode=false}
    \urlstyle{same}
                                


\begin{document}

\title{System Verification and Validation Plan for \progname{}} 
\author{\authname}
\date{\today}
	
\maketitle

\pagenumbering{roman}

\section*{Revision History}

\begin{tabularx}{\textwidth}{p{3cm}p{2cm}X}
\toprule {\bf Date} & {\bf Version} & {\bf Notes}\\
\midrule
Date 1 & 1.0 & Notes\\
Date 2 & 1.1 & Notes\\
\bottomrule
\end{tabularx}

~\\
\wss{The intention of the VnV plan is to increase confidence in the software.
However, this does not mean listing every verification and validation technique
that has ever been devised.  The VnV plan should also be a \textbf{feasible}
plan. Execution of the plan should be possible with the time and team available.
If the full plan cannot be completed during the time available, it can either be
modified to ``fake it'', or a better solution is to add a section describing
what work has been completed and what work is still planned for the future.}

\wss{The VnV plan is typically started after the requirements stage, but before
the design stage.  This means that the sections related to unit testing cannot
initially be completed.  The sections will be filled in after the design stage
is complete.  the final version of the VnV plan should have all sections filled
in.}

\newpage

\tableofcontents

\listoftables
\wss{Remove this section if it isn't needed}

\listoffigures
\wss{Remove this section if it isn't needed}

\newpage

\section{Symbols, Abbreviations, and Acronyms}

\renewcommand{\arraystretch}{1.2}
\begin{tabular}{l l} 
  \toprule		
  \textbf{symbol} & \textbf{description}\\
  \midrule 
  T & Test\\
  \bottomrule
\end{tabular}\\

\wss{symbols, abbreviations, or acronyms --- you can simply reference the SRS
  \citep{SRS} tables, if appropriate}

\wss{Remove this section if it isn't needed}

\newpage

\pagenumbering{arabic}

This document ... \wss{provide an introductory blurb and roadmap of the
  Verification and Validation plan}

\section{General Information}

\subsection{Summary}

\wss{Say what software is being tested.  Give its name and a brief overview of
  its general functions.}

\subsection{Objectives}

\wss{State what is intended to be accomplished.  The objective will be around
  the qualities that are most important for your project.  You might have
  something like: ``build confidence in the software correctness,''
  ``demonstrate adequate usability.'' etc.  You won't list all of the qualities,
  just those that are most important.}

\wss{You should also list the objectives that are out of scope.  You don't have 
the resources to do everything, so what will you be leaving out.  For instance, 
if you are not going to verify the quality of usability, state this.  It is also 
worthwhile to justify why the objectives are left out.}

\wss{The objectives are important because they highlight that you are aware of 
limitations in your resources for verification and validation.  You can't do everything, 
so what are you going to prioritize?  As an example, if your system depends on an 
external library, you can explicitly state that you will assume that external library 
has already been verified by its implementation team.}

\subsection{Challenge Level and Extras}

\wss{State the challenge level (advanced, general, basic) for your project.
Your challenge level should exactly match what is included in your problem
statement.  This should be the challenge level agreed on between you and the
course instructor.  You can use a pull request to update your challenge level
(in TeamComposition.csv or Repos.csv) if your plan changes as a result of the
VnV planning exercise.}

\wss{Summarize the extras (if any) that were tackled by this project.  Extras
can include usability testing, code walkthroughs, user documentation, formal
proof, GenderMag personas, Design Thinking, etc.  Extras should have already
been approved by the course instructor as included in your problem statement.
You can use a pull request to update your extras (in TeamComposition.csv or
Repos.csv) if your plan changes as a result of the VnV planning exercise.}

\subsection{Relevant Documentation}

\wss{Reference relevant documentation.  This will definitely include your SRS
  and your other project documents (design documents, like MG, MIS, etc).  You
  can include these even before they are written, since by the time the project
  is done, they will be written.  You can create BibTeX entries for your
  documents and within those entries include a hyperlink to the documents.}

\citet{SRS}

\wss{Don't just list the other documents.  You should explain why they are relevant and 
how they relate to your VnV efforts.}

\section{Plan}

\wss{Introduce this section.  You can provide a roadmap of the sections to
  come.}

\subsection{Verification and Validation Team}

\wss{Your teammates.  Maybe your supervisor.
  You should do more than list names.  You should say what each person's role is
  for the project's verification.  A table is a good way to summarize this information.}

\subsection{SRS Verification}

\wss{List any approaches you intend to use for SRS verification.  This may
  include ad hoc feedback from reviewers, like your classmates (like your
  primary reviewer), or you may plan for something more rigorous/systematic.}

\wss{If you have a supervisor for the project, you shouldn't just say they will
read over the SRS.  You should explain your structured approach to the review.
Will you have a meeting?  What will you present?  What questions will you ask?
Will you give them instructions for a task-based inspection?  Will you use your
issue tracker?}

\wss{Maybe create an SRS checklist?}

\subsection{Design Verification}

\wss{Plans for design verification}

\wss{The review will include reviews by your classmates}

\wss{Create a checklists?}

\subsection{Verification and Validation Plan Verification}

\wss{The verification and validation plan is an artifact that should also be
verified.  Techniques for this include review and mutation testing.}

\wss{The review will include reviews by your classmates}

\wss{Create a checklists?}

\subsection{Implementation Verification}

\wss{You should at least point to the tests listed in this document and the unit
  testing plan.}

\wss{In this section you would also give any details of any plans for static
  verification of the implementation.  Potential techniques include code
  walkthroughs, code inspection, static analyzers, etc.}

\wss{The final class presentation in CAS 741 could be used as a code
walkthrough.  There is also a possibility of using the final presentation (in
CAS741) for a partial usability survey.}

\subsection{Automated Testing and Verification Tools}

\wss{What tools are you using for automated testing.  Likely a unit testing
  framework and maybe a profiling tool, like ValGrind.  Other possible tools
  include a static analyzer, make, continuous integration tools, test coverage
  tools, etc.  Explain your plans for summarizing code coverage metrics.
  Linters are another important class of tools.  For the programming language
  you select, you should look at the available linters.  There may also be tools
  that verify that coding standards have been respected, like flake9 for
  Python.}

\wss{If you have already done this in the development plan, you can point to
that document.}

\wss{The details of this section will likely evolve as you get closer to the
  implementation.}

\subsection{Software Validation}

\wss{If there is any external data that can be used for validation, you should
  point to it here.  If there are no plans for validation, you should state that
  here.}

\wss{You might want to use review sessions with the stakeholder to check that
the requirements document captures the right requirements.  Maybe task based
inspection?}

\wss{For those capstone teams with an external supervisor, the Rev 0 demo should 
be used as an opportunity to validate the requirements.  You should plan on 
demonstrating your project to your supervisor shortly after the scheduled Rev 0 demo.  
The feedback from your supervisor will be very useful for improving your project.}

\wss{For teams without an external supervisor, user testing can serve the same purpose 
as a Rev 0 demo for the supervisor.}

\wss{This section might reference back to the SRS verification section.}

\section{System Tests}

\wss{There should be text between all headings, even if it is just a roadmap of
the contents of the subsections.}

\subsection{Tests for Functional Requirements}

\subsection{Tests for Nonfunctional Requirements - test}

\subsubsection{System Performance Testing}

\paragraph{Collection Processing Performance}

\begin{enumerate}

\item \textbf{test-nfr-perf-1}

\textbf{Type:} Dynamic, Automated, Performance
					
\textbf{Initial State:} System deployed with empty database; British Library collection dataset (21,000 fragments) prepared for ingestion
					
\textbf{Input/Condition:} Execute bulk fragment processing pipeline with full British Library collection; monitor processing time, resource utilization, and completion rate
					
\textbf{Output/Result:} Processing completes within 48 hours; all 21,000 fragments successfully processed with metadata extracted; system logs show no critical errors; database populated with fragment data
					
\textbf{How test will be performed:} Automated script will initiate the bulk processing workflow. Performance monitoring tools will track execution time, CPU usage, memory consumption, and I/O operations. Upon completion, validation queries will verify that all fragments are searchable with extracted metadata. Test will be considered passed if processing time ≤48 hours and success rate ≥99\%.

\item \textbf{test-nfr-perf-2}

\textbf{Type:} Dynamic, Manual, Load Testing
					
\textbf{Initial State:} System fully operational with British Library collection loaded; monitoring tools configured
					
\textbf{Input/Condition:} Simulate 15 concurrent users performing typical research workflows (searches, filtering, canvas operations) over a 60-minute period
					
\textbf{Output/Result:} System maintains responsiveness throughout test period; search queries return results within 3 seconds; canvas operations respond within 200ms; no user sessions experience timeouts or errors
					
\textbf{How test will be performed:} Load testing tool (e.g., Apache JMeter or Locust) will simulate 15 concurrent users executing randomized but realistic research workflows. Test scenarios include: fragment searches by ID and metadata filters, canvas workspace creation and manipulation, session save/restore operations. System performance metrics (response times, error rates, resource utilization) will be logged. Test passes if 95\% of operations meet specified response time thresholds and no critical errors occur.

\end{enumerate}

\subsubsection{AI Matching Assistance Quality}

\paragraph{Matching Suggestion Utility}

\begin{enumerate}

\item \textbf{test-nfr-ai-1}

\textbf{Type:} Dynamic, Manual, Expert Evaluation
					
\textbf{Initial State:} System operational with AI matching model trained and deployed; test set of 50 known fragment pairs prepared (25 true matches, 25 non-matches)
					
\textbf{Input/Condition:} For each test fragment, request AI matching suggestions; record top 10 suggestions and their confidence scores
					
\textbf{Output/Result:} Evaluation report showing precision, recall, and utility metrics; expert researchers rate suggestions as "useful for research workflow" in ≥60\% of cases
					
\textbf{How test will be performed:} Three Buddhist Studies researchers will independently evaluate AI matching suggestions for the 50-fragment test set. For each fragment, they will assess whether suggested matches merit scholarly investigation. Researchers will complete a standardized evaluation form rating suggestion utility on a 5-point scale and providing qualitative feedback. Results will be aggregated to determine if the 60\% utility threshold is met. Statistical analysis will include inter-rater reliability measures.

\item \textbf{test-nfr-ai-2}

\textbf{Type:} Functional, Dynamic, Automated
					
\textbf{Initial State:} AI model deployed; validation dataset of 100 fragment pairs with ground truth labels available
					
\textbf{Input/Condition:} Execute batch prediction on validation dataset; compare model predictions against ground truth
					
\textbf{Output/Result:} Model performance metrics calculated: precision ≥70\%, recall ≥65\%, F1-score ≥67\%; confusion matrix and ROC curve generated for analysis
					
\textbf{How test will be performed:} Automated test script will process the validation dataset through the AI matching pipeline. Predictions will be compared against ground truth labels to calculate standard classification metrics. Results will be documented in a test report including: confusion matrix, precision-recall curves, distribution of confidence scores, and analysis of false positives/negatives. Test passes if performance meets or exceeds specified thresholds.

\end{enumerate}

\subsubsection{Usability Testing}

\paragraph{Research Workflow Task Completion}

\begin{enumerate}

\item \textbf{test-nfr-usability-1}

\textbf{Type:} Dynamic, Manual, User Testing
					
\textbf{Initial State:} System fully operational; test participants (5-7 Buddhist Studies researchers) recruited; standardized task scenarios prepared
					
\textbf{Input/Condition:} Each participant performs 8-10 representative research tasks: searching fragments by metadata, applying multi-criteria filters, creating canvas workspaces, arranging fragments, saving and restoring sessions
					
\textbf{Output/Result:} Task completion rate ≥80\% across all participants and tasks; average time-on-task within acceptable ranges; System Usability Scale (SUS) score ≥70
					
\textbf{How test will be performed:} Participants will complete tasks in a moderated usability testing session (60-90 minutes). Sessions will be recorded (with consent) for later analysis. For each task, observers will record: completion status (success/fail), time-on-task, number of errors, and assistance required. Post-test, participants will complete the SUS questionnaire and a custom satisfaction survey. Qualitative feedback will be collected through semi-structured interviews. Test passes if overall task completion rate ≥80\% and SUS score ≥70.

\item \textbf{test-nfr-usability-2}

\textbf{Type:} Static, Manual, Heuristic Evaluation
					
\textbf{Initial State:} System interface fully implemented and accessible for evaluation
					
\textbf{Input/Condition:} Three UX experts independently evaluate the interface against Nielsen's 10 usability heuristics
					
\textbf{Output/Result:} Consolidated report identifying usability issues categorized by severity (critical, major, minor); prioritized list of recommended improvements
					
\textbf{How test will be performed:} Each evaluator will spend 2-3 hours exploring the interface and documenting violations of usability heuristics. Issues will be rated by severity (0-4 scale) and likelihood of occurrence. Evaluators will meet to consolidate findings, resolve disagreements, and prioritize issues. A final report will document all identified issues with recommendations. Critical and major issues must be addressed before deployment; minor issues will be tracked for future iterations.

\end{enumerate}

\subsubsection{System Availability and Reliability}

\paragraph{Uptime and Availability Testing}

\begin{enumerate}

\item \textbf{test-nfr-reliability-1}

\textbf{Type:} Dynamic, Automated, Reliability Testing
					
\textbf{Initial State:} System deployed to production environment; monitoring and alerting configured
					
\textbf{Input/Condition:} Continuous operation over 30-day period with automated health checks every 5 minutes; simulated user activity maintained throughout period
					
\textbf{Output/Result:} System availability ≥95\% during operational hours (defined as 8:00 AM - 10:00 PM Eastern Time, Monday-Sunday); mean time between failures (MTBF) ≥168 hours; mean time to recovery (MTTR) ≤30 minutes for non-critical failures
					
\textbf{How test will be performed:} Monitoring system (e.g., Prometheus with Grafana) will track system availability, response times, and error rates. Automated health checks will verify core functionality: API endpoints responding, database connectivity, search operations, authentication services. All downtime incidents will be logged with root cause analysis. Monthly availability report will be generated showing uptime percentage, incident summary, and adherence to service level objectives. Test is continuous and evaluated monthly.

\item \textbf{test-nfr-reliability-2}

\textbf{Type:} Dynamic, Manual, Disaster Recovery
					
\textbf{Initial State:} System operational with full dataset; backup procedures configured
					
\textbf{Input/Condition:} Simulate critical failure scenario (e.g., database corruption, server failure); initiate disaster recovery procedures
					
\textbf{Output/Result:} System restored to operational state within 4 hours; no data loss for user sessions and collections; all functionality verified post-recovery
					
\textbf{How test will be performed:} Disaster recovery test will be conducted in a controlled manner during scheduled maintenance window. Test will simulate a critical failure, and the team will execute documented recovery procedures. Recovery time will be measured from failure detection to full service restoration. Post-recovery verification will include: database integrity checks, search functionality validation, user session restoration, and comparison of dataset checksums. Test documentation will include timeline of recovery actions and lessons learned.

\end{enumerate}

\subsubsection{Data Integrity and Security}

\paragraph{Data Preservation and Attribution}

\begin{enumerate}

\item \textbf{test-nfr-data-1}

\textbf{Type:} Functional, Automated, Data Validation
					
\textbf{Initial State:} British Library collection metadata and user-submitted images loaded into system
					
\textbf{Input/Condition:} Execute comprehensive data validation checks across entire collection
					
\textbf{Output/Result:} 100\% of fragments retain original British Library catalog information (shelfmark, collection, provenance); all user-submitted images maintain uploaded attribution; no corruption or loss of metadata fields
					
\textbf{How test will be performed:} Automated validation script will query all fragments in the database and verify: (1) presence of required metadata fields, (2) consistency with source British Library catalog data, (3) preservation of user attribution for uploaded images, (4) referential integrity of relational data. Script will generate exception report for any discrepancies. Manual spot-checking of 100 randomly selected fragments will verify accuracy. Test passes with 100\% preservation of provenance and catalog information.

\item \textbf{test-nfr-data-2}

\textbf{Type:} Static, Manual, Code Review
					
\textbf{Initial State:} Data handling code modules identified for review; reviewers assigned
					
\textbf{Input/Condition:} Security-focused code review of data handling, authentication, and authorization modules
					
\textbf{Output/Result:} Review checklist completed; no critical security vulnerabilities identified; best practices for data protection verified
					
\textbf{How test will be performed:} Two senior developers will conduct a thorough code review focusing on: input validation and sanitization, SQL injection prevention, authentication mechanisms, authorization controls, data encryption at rest and in transit, and secure API design. Review will use OWASP security guidelines as reference. All findings will be documented with severity ratings. Critical and high-severity issues must be resolved before deployment. Review documentation will be maintained for audit purposes.

\end{enumerate}

\subsubsection{Scalability Testing}

\paragraph{Fragment Collection Scalability}

\begin{enumerate}

\item \textbf{test-nfr-scale-1}

\textbf{Type:} Dynamic, Automated, Stress Testing
					
\textbf{Initial State:} System operational with British Library collection (21,000 fragments)
					
\textbf{Input/Condition:} Incrementally add synthetic fragment data to test system behavior at 50,000, 100,000, and 200,000 fragments
					
\textbf{Output/Result:} Performance degradation versus collection size documented in summary table and graphs; search response time increases no more than linearly with collection size; system remains stable at 200,000 fragments
					
\textbf{How test will be performed:} Synthetic fragment data will be generated with realistic metadata distributions. For each collection size, standardized performance tests will be executed: simple searches, complex multi-filter queries, and canvas operations. Response times, memory usage, and query execution plans will be recorded. Results will be presented as tables and graphs showing performance metrics versus collection size. Test provides quantitative data for future scaling decisions rather than pass/fail criteria.

\end{enumerate}

\subsubsection{Cross-Browser and Device Compatibility}

\paragraph{Interface Compatibility Testing}

\begin{enumerate}

\item \textbf{test-nfr-compat-1}

\textbf{Type:} Manual, Dynamic, Compatibility Testing
					
\textbf{Initial State:} System deployed and accessible via URL
					
\textbf{Input/Condition:} Access system and perform core workflows on multiple browser and device combinations: Chrome, Firefox, Safari (desktop); Chrome, Safari (mobile); tablets
					
\textbf{Output/Result:} All core functionality (search, filter, canvas) operates correctly across tested platforms; visual layout renders appropriately; no critical browser-specific bugs identified
					
\textbf{How test will be performed:} Test team will execute a standardized test suite covering critical user paths on each browser/device combination. Testing will verify: page rendering, responsive design behavior, drag-and-drop functionality, session persistence, image loading. Cross-browser testing tool (e.g., BrowserStack) may be used to expand coverage. Issues will be documented with severity ratings and browser-specific details. Critical issues preventing core functionality must be resolved; minor visual inconsistencies will be tracked for future improvements.

\end{enumerate}

\subsubsection{Performance Benchmarking}

\paragraph{Response Time Measurement}

\begin{enumerate}

\item \textbf{test-nfr-bench-1}

\textbf{Type:} Dynamic, Automated, Benchmark Testing
					
\textbf{Initial State:} System operational with full British Library collection
					
\textbf{Input/Condition:} Execute standardized set of operations with varying complexity and problem sizes: simple fragment ID lookups, 1-criterion filters, 3-criterion filters, 5-criterion filters, canvas operations with 10, 50, 100 fragments
					
\textbf{Output/Result:} Comprehensive performance report documenting response times for each operation type and complexity level; summary table showing min, max, mean, median, and 95th percentile response times; performance graphs illustrating trends
					
\textbf{How test will be performed:} Automated benchmark suite will execute each operation type 100 times, recording response times. Tests will run during low-activity periods to establish baseline performance. Results will be analyzed statistically, removing outliers. Performance data will be presented in summary tables and visualization graphs (response time vs. operation complexity, response time distributions). This benchmark establishes performance baselines for regression testing and capacity planning rather than enforcing rigid pass/fail thresholds.

\end{enumerate}

...

\subsection{Traceability Between Test Cases and Requirements}

\wss{Provide a table that shows which test cases are supporting which
  requirements.}

\section{Unit Test Description}

\wss{This section should not be filled in until after the MIS (detailed design
  document) has been completed.}

\wss{Reference your MIS (detailed design document) and explain your overall
philosophy for test case selection.}  

\wss{To save space and time, it may be an option to provide less detail in this section.  
For the unit tests you can potentially layout your testing strategy here.  That is, you 
can explain how tests will be selected for each module.  For instance, your test building 
approach could be test cases for each access program, including one test for normal behaviour 
and as many tests as needed for edge cases.  Rather than create the details of the input 
and output here, you could point to the unit testing code.  For this to work, you code 
needs to be well-documented, with meaningful names for all of the tests.}

\subsection{Unit Testing Scope}

\wss{What modules are outside of the scope.  If there are modules that are
  developed by someone else, then you would say here if you aren't planning on
  verifying them.  There may also be modules that are part of your software, but
  have a lower priority for verification than others.  If this is the case,
  explain your rationale for the ranking of module importance.}

\subsection{Tests for Functional Requirements}

\wss{Most of the verification will be through automated unit testing.  If
  appropriate specific modules can be verified by a non-testing based
  technique.  That can also be documented in this section.}

\subsubsection{Module 1}

\wss{Include a blurb here to explain why the subsections below cover the module.
  References to the MIS would be good.  You will want tests from a black box
  perspective and from a white box perspective.  Explain to the reader how the
  tests were selected.}

\begin{enumerate}

\item{test-id1\\}

Type: \wss{Functional, Dynamic, Manual, Automatic, Static etc. Most will
  be automatic}
					
Initial State: 
					
Input: 
					
Output: \wss{The expected result for the given inputs}

Test Case Derivation: \wss{Justify the expected value given in the Output field}

How test will be performed: 
					
\item{test-id2\\}

Type: \wss{Functional, Dynamic, Manual, Automatic, Static etc. Most will
  be automatic}
					
Initial State: 
					
Input: 
					
Output: \wss{The expected result for the given inputs}

Test Case Derivation: \wss{Justify the expected value given in the Output field}

How test will be performed: 

\item{...\\}
    
\end{enumerate}

\subsubsection{Module 2}

...

\subsection{Tests for Nonfunctional Requirements}

\wss{If there is a module that needs to be independently assessed for
  performance, those test cases can go here.  In some projects, planning for
  nonfunctional tests of units will not be that relevant.}

\wss{These tests may involve collecting performance data from previously
  mentioned functional tests.}

\subsubsection{Module ?}
		
\begin{enumerate}

\item{test-id1\\}

Type: \wss{Functional, Dynamic, Manual, Automatic, Static etc. Most will
  be automatic}
					
Initial State: 
					
Input/Condition: 
					
Output/Result: 
					
How test will be performed: 
					
\item{test-id2\\}

Type: Functional, Dynamic, Manual, Static etc.
					
Initial State: 
					
Input: 
					
Output: 
					
How test will be performed: 

\end{enumerate}

\subsubsection{Module ?}

...

\subsection{Traceability Between Test Cases and Modules}

\wss{Provide evidence that all of the modules have been considered.}
				
\bibliographystyle{plainnat}

\bibliography{../../refs/References}

\newpage

\section{Appendix}

This is where you can place additional information.

\subsection{Symbolic Parameters}

The definition of the test cases will call for SYMBOLIC\_CONSTANTS.
Their values are defined in this section for easy maintenance.

\subsection{Usability Survey Questions?}

\wss{This is a section that would be appropriate for some projects.}

\newpage{}
\section*{Appendix --- Reflection}

\wss{This section is not required for CAS 741}

The information in this section will be used to evaluate the team members on the
graduate attribute of Lifelong Learning.

The purpose of reflection questions is to give you a chance to assess your own
learning and that of your group as a whole, and to find ways to improve in the
future. Reflection is an important part of the learning process.  Reflection is
also an essential component of a successful software development process.  

Reflections are most interesting and useful when they're honest, even if the
stories they tell are imperfect. You will be marked based on your depth of
thought and analysis, and not based on the content of the reflections
themselves. Thus, for full marks we encourage you to answer openly and honestly
and to avoid simply writing ``what you think the evaluator wants to hear.''

Please answer the following questions.  Some questions can be answered on the
team level, but where appropriate, each team member should write their own
response:


\begin{enumerate}
  \item What went well while writing this deliverable? 
  \item What pain points did you experience during this deliverable, and how
    did you resolve them?
  \item What knowledge and skills will the team collectively need to acquire to
  successfully complete the verification and validation of your project?
  Examples of possible knowledge and skills include dynamic testing knowledge,
  static testing knowledge, specific tool usage, Valgrind etc.  You should look to
  identify at least one item for each team member.
  \item For each of the knowledge areas and skills identified in the previous
  question, what are at least two approaches to acquiring the knowledge or
  mastering the skill?  Of the identified approaches, which will each team
  member pursue, and why did they make this choice?
\end{enumerate}

\end{document}