\documentclass{article}

\usepackage{booktabs}
\usepackage{tabularx}
\usepackage{hyperref}

\hypersetup{
    colorlinks=true,       % false: boxed links; true: colored links
    linkcolor=red,          % color of internal links (change box color with linkbordercolor)
    citecolor=green,        % color of links to bibliography
    filecolor=magenta,      % color of file links
    urlcolor=cyan           % color of external links
}

\title{Hazard Analysis\\\progname}

\author{\authname}

\date{}

\input{../Comments}
%% Common Parts

\newcommand{\progname}{Software Engineering} % PUT YOUR PROGRAM NAME HERE
\newcommand{\authname}{Team \#1, Sanskrit Ciphers
\\ Omar El Aref
\\ Dylan Garner
\\ Muhammad Umar Khan
\\ Aswin Kuganesan
\\ Yousef Shahin} % AUTHOR NAMES                  

\usepackage{hyperref}
    \hypersetup{colorlinks=true, linkcolor=blue, citecolor=blue, filecolor=blue,
                urlcolor=blue, unicode=false}
    \urlstyle{same}
                                


\begin{document}

\maketitle
\thispagestyle{empty}

~\newpage

\pagenumbering{roman}

\begin{table}[hp]
\caption{Revision History} \label{TblRevisionHistory}
\begin{tabularx}{\textwidth}{llX}
\toprule
\textbf{Date} & \textbf{Developer(s)} & \textbf{Change}\\
\midrule
Date1 & Name(s) & Description of changes\\
Date2 & Name(s) & Description of changes\\
... & ... & ...\\
\bottomrule
\end{tabularx}
\end{table}

~\newpage

\tableofcontents

~\newpage

\pagenumbering{arabic}

\wss{You are free to modify this template.}

\section{Introduction}

\subsection{Definition of Hazard}

Following the definition from hazard analysis literature, a \textbf{hazard} is defined as a property or condition in the Sanskrit Manuscript Fragment Reconstruction Platform system together with a condition in the environment that has the potential to cause harm or damage (loss). In the context of this academic research tool, harm extends beyond traditional safety concerns to include:

\begin{itemize}
\item \textbf{Security hazards}: Unauthorized access to cultural heritage data
\item \textbf{Data quality hazards}: Corruption or loss of manuscript information
\item \textbf{Research integrity hazards}: False scholarly conclusions from AI suggestions
\item \textbf{Usability hazards}: Frustrating or ineffective research workflows
\item \textbf{Cultural sensitivity hazards}: Inappropriate handling of sacred materials
\item \textbf{Software quality hazards}: System crashes or unreliable performance
\end{itemize}

\subsection{Project Context and Unique Hazard Considerations}

The Sanskrit Manuscript Fragment Reconstruction Platform operates within a distinctive environment that creates unique hazard profiles not typically addressed in conventional software projects:

\subsubsection{Cultural Heritage Sensitivity}
The system handles irreplaceable cultural artifacts representing centuries of Buddhist scholarship and religious tradition. Any computational error or inappropriate handling could potentially damage scholarly understanding of these materials or violate cultural sensitivities. Unlike commercial software where errors may result in financial loss, errors in this system could result in irreversible harm to cultural heritage knowledge.

\subsubsection{Academic Research Integrity}
The platform directly influences scholarly research outcomes and academic publications. AI-generated suggestions carry the risk of being accepted without sufficient critical evaluation, potentially leading to false scholarly conclusions that could propagate through academic literature for decades. The traditional peer review system may be inadequately prepared to evaluate AI-assisted research, creating novel risks to research quality.

\subsubsection{Non-Critical AI Assistance Context}
As an AI-powered research assistance tool, the platform operates in a non-critical domain where the primary risks involve inefficient scholarly time use rather than immediate safety threats. The non-deterministic nature of AI suggestions requires that all outputs be clearly presented as requiring scholarly evaluation rather than definitive answers.

\subsection{Scope of Hazard Analysis}

This hazard analysis addresses potential risks arising from:

\begin{enumerate}
\item \textbf{Direct System Operation}: Hazards emerging from normal use of the platform's core functionality
\item \textbf{User Interaction Patterns}: Risks created by how scholars interact with AI-assisted tools
\item \textbf{Data Handling and Processing}: Hazards related to the management of sensitive cultural heritage data
\item \textbf{Long-term System Evolution}: Hazards that may develop over time as the platform matures and usage patterns evolve
\item \textbf{Integration with Existing Systems}: Risks created by connecting the platform with institutional databases and archives
\end{enumerate}

\section{Scope and Purpose of Hazard Analysis}

\subsection{Purpose}

This hazard analysis identifies and evaluates potential risks associated with the Sanskrit Manuscript Fragment Reconstruction Platform to ensure safe, responsible deployment in academic research environments. The analysis focuses on protecting cultural heritage materials, maintaining scholarly integrity, and preventing negative impacts on Buddhist Studies research.

\subsection{Potential Losses}

The following categories of loss could be incurred due to identified hazards:

\subsubsection{Cultural Heritage Losses}
\begin{itemize}
\item \textbf{Irreversible damage to scholarly understanding} of ancient Buddhist texts through computational errors or misinterpretation
\item \textbf{Loss of cultural sensitivity} in handling sacred or culturally significant manuscript materials
\item \textbf{Degradation of traditional scholarly skills} in paleography and manuscript analysis due to over-reliance on computational tools
\end{itemize}

\subsubsection{Academic and Research Integrity Losses}
\begin{itemize}
\item \textbf{Publication of false scholarly conclusions} based on uncritically accepted AI suggestions, potentially propagating through academic literature
\item \textbf{Erosion of research quality standards} if AI-assisted research bypasses traditional peer review processes
\item \textbf{Loss of research reproducibility} due to non-deterministic AI outputs and inadequate documentation
\item \textbf{Compromise of scholarly attribution} and academic credit for discoveries
\end{itemize}

\section{System Boundaries and Components}

\wss{Dividing the system into components will help you brainstorm the hazards.
You shouldn't do a full design of the components, just get a feel for the major
ones.  For projects that involve hardware, the components will typically include
each individual piece of hardware.  If your software will have a database, or an
important library, these are also potential components.}

\section{Critical Assumptions}

\wss{These assumptions that are made about the software or system.  You should
minimize the number of assumptions that remove potential hazards.  For instance,
you could assume a part will never fail, but it is generally better to include
this potential failure mode.}

\section{Failure Mode and Effect Analysis}

\wss{Include your FMEA table here. This is the most important part of this document.}
\wss{The safety requirements in the table do not have to have the prefix SR.
The most important thing is to show traceability to your SRS. You might trace to
requirements you have already written, or you might need to add new
requirements.}
\wss{If no safety requirement can be devised, other mitigation strategies can be
entered in the table, including strategies involving providing additional
documentation, and/or test cases.}

\section{Safety and Security Requirements}

\wss{Newly discovered requirements.  These should also be added to the SRS.  (A
rationale design process how and why to fake it.)}

\section{Roadmap}

\wss{Which safety requirements will be implemented as part of the capstone timeline?
Which requirements will be implemented in the future?}

\newpage{}

\section*{Appendix --- Reflection}

\wss{Not required for CAS 741}

\input{../Reflection.tex}

\begin{enumerate}
    \item What went well while writing this deliverable? 
    \item What pain points did you experience during this deliverable, and how
    did you resolve them?
    \item Which of your listed risks had your team thought of before this
    deliverable, and which did you think of while doing this deliverable? For
    the latter ones (ones you thought of while doing the Hazard Analysis), how
    did they come about?
    \item Other than the risk of physical harm (some projects may not have any
    appreciable risks of this form), list at least 2 other types of risk in
    software products. Why are they important to consider?
\end{enumerate}

\end{document}