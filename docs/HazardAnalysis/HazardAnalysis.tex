\documentclass{article}

\usepackage{booktabs}
\usepackage{tabularx}
\usepackage{hyperref}

\hypersetup{
    colorlinks=true,       % false: boxed links; true: colored links
    linkcolor=red,          % color of internal links (change box color with linkbordercolor)
    citecolor=green,        % color of links to bibliography
    filecolor=magenta,      % color of file links
    urlcolor=cyan           % color of external links
}

\title{Hazard Analysis\\\progname}

\author{\authname}

\date{}

\input{../Comments}
%% Common Parts

\newcommand{\progname}{Software Engineering} % PUT YOUR PROGRAM NAME HERE
\newcommand{\authname}{Team \#1, Sanskrit Ciphers
\\ Omar El Aref
\\ Dylan Garner
\\ Muhammad Umar Khan
\\ Aswin Kuganesan
\\ Yousef Shahin} % AUTHOR NAMES                  

\usepackage{hyperref}
    \hypersetup{colorlinks=true, linkcolor=blue, citecolor=blue, filecolor=blue,
                urlcolor=blue, unicode=false}
    \urlstyle{same}
                                


\begin{document}

\maketitle
\thispagestyle{empty}

~\newpage

\pagenumbering{roman}

\begin{table}[hp]
\caption{Revision History} \label{TblRevisionHistory}
\begin{tabularx}{\textwidth}{llX}
\toprule
\textbf{Date} & \textbf{Developer(s)} & \textbf{Change}\\
\midrule
Date1 & Name(s) & Description of changes\\
Date2 & Name(s) & Description of changes\\
... & ... & ...\\
\bottomrule
\end{tabularx}
\end{table}

~\newpage

\tableofcontents

~\newpage

\pagenumbering{arabic}

\wss{You are free to modify this template.}

\section{Introduction}

\wss{You can include your definition of what a hazard is here.}

\section{Scope and Purpose of Hazard Analysis}

\wss{You should say what \textbf{loss} could be incurred because of the
hazards.}

\section{System Boundaries and Components}

\wss{Dividing the system into components will help you brainstorm the hazards.
You shouldn't do a full design of the components, just get a feel for the major
ones.  For projects that involve hardware, the components will typically include
each individual piece of hardware.  If your software will have a database, or an
important library, these are also potential components.}

\section{Critical Assumptions}

\begin{itemize}
    \item Input images are provided in a usable format for the system and are not corrupted when uploaded
    \item Resolution of fragment images is sufficient for meaningful feature extraction using edges and damage patterns
    \item Users of the system are expected to understand the basic functionality of the system and be knowledgeable in religious studies
    \item External APIs and libraries used (image processing/segmentation) will be supported throughout the entire project lifetime.
    \item All images uploaded and stored in the database are images of ancient fragments that the user has been permitted to use
\end{itemize}

\section{Failure Mode and Effect Analysis}

\begin{table}[htbp]
\centering
\scriptsize
\caption{Failure Mode and Effects Analysis (FMEA) for Manuscript Recreation System}
\begin{tabularx}{\textwidth}{p{2.5cm} X X X X}
\toprule
\textbf{Design Function} & \textbf{Failure Modes} & \textbf{Effects of Failure} & \textbf{Causes of Failure} & \textbf{Recommended Action} \\
\midrule

\multirow{7}{*}{\parbox[t]{2cm}{Manuscript \\ Recreation}}
& Image upload fails or corrupted file is accepted 
& Missing or unusable fragment in display
& Invalid file format, incomplete upload, or corrupted image data 
& Validate file format during upload and provide error messages to users \\

& Misalignment tools produce incorrect rotation/zoom 
& User arranges fragments inaccurately, leading to wrong conclusions 
& Calibration error or algorithmic inaccuracy in transformation logic 
& Allow users to undo and redo previous action \\

& Unauthorized user gains access 
& Data breach or unauthorized edits to manuscripts 
& Weak password, unencrypted session, or bypassed authentication 
& Enforce strong passwords and 2 factor authentication \\

& Data corruption or loss 
& User progress and annotations are lost 
& Database write failure or transaction interruption 
& Perform daily backups, implement transaction rollback and recovery mechanisms \\

& False positives in edge similarity 
& Incorrect fragment matches suggested to user 
& Overfitting or poor training data distribution 
& Require human validation, display probability percentages and match explanations \\

& User loses connection during upload or save 
& Session progress lost 
& Network interruption or unstable client connection 
& Autosave state locally, synchronize data once the connection is restored \\

\bottomrule
\end{tabularx}
\end{table}

\section{Safety and Security Requirements}

\wss{Newly discovered requirements.  These should also be added to the SRS.  (A
rationale design process how and why to fake it.)}

\section{Roadmap}

\wss{Which safety requirements will be implemented as part of the capstone timeline?
Which requirements will be implemented in the future?}

\newpage{}

\section*{Appendix --- Reflection}

\wss{Not required for CAS 741}

\input{../Reflection.tex}

\begin{enumerate}
    \item What went well while writing this deliverable? 
    \item What pain points did you experience during this deliverable, and how
    did you resolve them?
    \item Which of your listed risks had your team thought of before this
    deliverable, and which did you think of while doing this deliverable? For
    the latter ones (ones you thought of while doing the Hazard Analysis), how
    did they come about?
    \item Other than the risk of physical harm (some projects may not have any
    appreciable risks of this form), list at least 2 other types of risk in
    software products. Why are they important to consider?
\end{enumerate}

\end{document}