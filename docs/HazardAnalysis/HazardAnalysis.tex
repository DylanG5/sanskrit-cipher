\documentclass{article}

\usepackage{booktabs}
\usepackage{tabularx}
\usepackage{hyperref}

\hypersetup{
    colorlinks=true,       % false: boxed links; true: colored links
    linkcolor=red,          % color of internal links (change box color with linkbordercolor)
    citecolor=green,        % color of links to bibliography
    filecolor=magenta,      % color of file links
    urlcolor=cyan           % color of external links
}

\title{Hazard Analysis\\\progname}

\author{\authname}

\date{}

%% Comments

\usepackage{color}

\newif\ifcomments\commentstrue %displays comments
%\newif\ifcomments\commentsfalse %so that comments do not display

\ifcomments
\newcommand{\authornote}[3]{\textcolor{#1}{[#3 ---#2]}}
\newcommand{\todo}[1]{\textcolor{red}{[TODO: #1]}}
\else
\newcommand{\authornote}[3]{}
\newcommand{\todo}[1]{}
\fi

\newcommand{\wss}[1]{\authornote{magenta}{SS}{#1}} 
\newcommand{\plt}[1]{\authornote{cyan}{TPLT}{#1}} %For explanation of the template
\newcommand{\an}[1]{\authornote{cyan}{Author}{#1}}

%% Common Parts

\newcommand{\progname}{ProgName} % PUT YOUR PROGRAM NAME HERE
\newcommand{\authname}{Team \#, Team Name
\\ Student 1 name
\\ Student 2 name
\\ Student 3 name
\\ Student 4 name} % AUTHOR NAMES                  

\usepackage{hyperref}
    \hypersetup{colorlinks=true, linkcolor=blue, citecolor=blue, filecolor=blue,
                urlcolor=blue, unicode=false}
    \urlstyle{same}
                                


\begin{document}

\maketitle
\thispagestyle{empty}

~\newpage

\pagenumbering{roman}

\begin{table}[hp]
\caption{Revision History} \label{TblRevisionHistory}
\begin{tabularx}{\textwidth}{llX}
\toprule
\textbf{Date} & \textbf{Developer(s)} & \textbf{Change}\\
\midrule
Date1 & Name(s) & Description of changes\\
Date2 & Name(s) & Description of changes\\
... & ... & ...\\
\bottomrule
\end{tabularx}
\end{table}

~\newpage

\tableofcontents

~\newpage

\pagenumbering{arabic}

\wss{You are free to modify this template.}

\section{Introduction}

\subsection{Definition of Hazard}

Following the definition from hazard analysis literature, a \textbf{hazard} is defined as a property or condition in the Sanskrit Manuscript Fragment Reconstruction Platform system together with a condition in the environment that has the potential to cause harm or damage (loss).

\subsubsection{System Conditions}

System conditions are properties or states of the software that create vulnerability or risk. Examples include:

\begin{itemize}
\item \textbf{Vulnerable database configuration}: Database exposed to unauthorized access attempts
\item \textbf{Inaccurate Artificial Intelligence (AI) models}: Script classification model with low accuracy (<50\%)
\item \textbf{Insufficient input validation}: System accepts malformed or malicious data
\item \textbf{Poor error handling}: System crashes or enters unstable state on unexpected input
\item \textbf{Inadequate session management}: Sessions persist beyond intended duration or lack proper isolation
\item \textbf{Resource allocation issues}: Memory leaks, unbounded queries, or inefficient algorithms
\item \textbf{Missing accessibility features}: Interface elements without keyboard navigation or screen reader support
\end{itemize}

\subsubsection{Environmental Conditions}

Environmental conditions are external circumstances or user behaviors that, when combined with system conditions, can cause harm. Examples include:

\begin{itemize}
\item \textbf{User attempts unauthorized data access}: User tries to download entire database or access restricted content
\item \textbf{User trusts inaccurate AI predictions}: Scholar bases research conclusions on low-confidence model outputs
\item \textbf{Network interruptions during critical operations}: Connection lost while saving research data
\item \textbf{Concurrent user access to shared resources}: Multiple researchers editing same fragment simultaneously
\item \textbf{Users with varying technical expertise}: Non-technical users attempting complex system operations
\item \textbf{Users with accessibility needs}: Users requiring assistive technologies to interact with system
\end{itemize}

\subsection{Project Context and Unique Hazard Considerations}

The Sanskrit Manuscript Fragment Reconstruction Platform operates within a distinctive environment that creates unique user experience hazards not typically addressed in conventional software projects:

\subsubsection{Diverse User Base with Varying Technical Skills}
The system serves users ranging from graduate students learning paleography to expert scholars to archival staff, each with different technical comfort levels and research needs. Poor interface design or insensitive feature implementation could exclude or frustrate significant user groups, leading to abandonment of the tool or ineffective research workflows.

\subsubsection{Research Workflow Integration Challenges}
Scholars have established research methodologies and workspace preferences developed over years of practice. Software that disrupts familiar workflows, lacks intuitive navigation, or forces users to adapt to rigid system constraints can cause significant frustration and reduce research productivity. Poor performance or unreliable functionality breaks concentration and research flow.

\subsubsection{Data Privacy and Confidentiality Concerns}
Researchers work with sensitive data including unpublished discoveries, institutional collaborations, and potentially restricted manuscript materials. Unintended data exposure through system vulnerabilities, poor session management, or inadequate access controls could compromise ongoing research, violate institutional agreements, or expose confidential scholarly work to competitors.


\section{Scope and Purpose of Hazard Analysis}

\subsection{Purpose}

This hazard analysis identifies and evaluates potential risks associated with the Sanskrit Manuscript Fragment Reconstruction Platform to ensure safe, responsible deployment in academic research environments. The analysis focuses on protecting cultural heritage materials, maintaining scholarly integrity, and preventing negative impacts on Buddhist Studies research. Through systematic verification testing and validation processes, this analysis helps identify potential system failures and ensures that main functionalities operate safely and reliably before deployment.

\subsection{Potential Losses}

Table~\ref{TblPotentialLosses} summarizes the potential losses that could occur from identified hazards in the system. These losses span data security, user experience, research productivity, and system reliability concerns.

\begin{table}[h]
\caption{Potential Losses from System Hazards} \label{TblPotentialLosses}
\begin{tabularx}{\textwidth}{lX}
\toprule
\textbf{Loss Category} & \textbf{Specific Losses}\\
\midrule
\textbf{Data Security} &
\begin{itemize}
\item Exposure of confidential research data
\item Unauthorized access to manuscript database
\item Unintended data sharing between users
\item Violation of institutional data agreements
\item Loss of competitive research advantage
\end{itemize}\\
\midrule
\textbf{Research Productivity} &
\begin{itemize}
\item Lost research time due to crashes or slow performance
\item Workflow disruption from poor interface design
\item Research momentum loss requiring work restart
\item Inefficient task completion from confusing navigation
\end{itemize}\\
\midrule
\textbf{Data Integrity} &
\begin{itemize}
\item Data corruption compromising months of work
\item Inadvertent modification of research data
\item Loss of unsaved work due to system failures
\item Incorrect fragment matches from inaccurate AI
\end{itemize}\\
\midrule
\textbf{User Experience} &
\begin{itemize}
\item System unresponsiveness or freezing
\item Frequent errors reducing user confidence
\item Frustration from overly complex interfaces
\item Inconsistent behavior across browsers/devices
\end{itemize}\\
\midrule
\textbf{Accessibility \& Inclusion} &
\begin{itemize}
\item Exclusion of users with disabilities
\item Frustration for non-technical users
\item Reduced adoption by diverse user groups
\item User abandonment from steep learning curve
\end{itemize}\\
\bottomrule
\end{tabularx}
\end{table}

\section{System Boundaries and Components}

\wss{Dividing the system into components will help you brainstorm the hazards.
You shouldn't do a full design of the components, just get a feel for the major
ones.  For projects that involve hardware, the components will typically include
each individual piece of hardware.  If your software will have a database, or an
important library, these are also potential components.}

\section{Critical Assumptions}

\wss{These assumptions that are made about the software or system.  You should
minimize the number of assumptions that remove potential hazards.  For instance,
you could assume a part will never fail, but it is generally better to include
this potential failure mode.}

\section{Failure Mode and Effect Analysis}

\wss{Include your FMEA table here. This is the most important part of this document.}
\wss{The safety requirements in the table do not have to have the prefix SR.
The most important thing is to show traceability to your SRS. You might trace to
requirements you have already written, or you might need to add new
requirements.}
\wss{If no safety requirement can be devised, other mitigation strategies can be
entered in the table, including strategies involving providing additional
documentation, and/or test cases.}

\section{Safety and Security Requirements}

\wss{Newly discovered requirements.  These should also be added to the SRS.  (A
rationale design process how and why to fake it.)}

\section{Roadmap}

\wss{Which safety requirements will be implemented as part of the capstone timeline?
Which requirements will be implemented in the future?}

\newpage{}

\section*{Appendix --- Reflection}

\wss{Not required for CAS 741}

The purpose of reflection questions is to give you a chance to assess your own
learning and that of your group as a whole, and to find ways to improve in the
future. Reflection is an important part of the learning process.  Reflection is
also an essential component of a successful software development process.  

Reflections are most interesting and useful when they're honest, even if the
stories they tell are imperfect. You will be marked based on your depth of
thought and analysis, and not based on the content of the reflections
themselves. Thus, for full marks we encourage you to answer openly and honestly
and to avoid simply writing ``what you think the evaluator wants to hear.''

Please answer the following questions.  Some questions can be answered on the
team level, but where appropriate, each team member should write their own
response:


\begin{enumerate}
    \item What went well while writing this deliverable? 
    \item What pain points did you experience during this deliverable, and how
    did you resolve them?
    \item Which of your listed risks had your team thought of before this
    deliverable, and which did you think of while doing this deliverable? For
    the latter ones (ones you thought of while doing the Hazard Analysis), how
    did they come about?
    \item Other than the risk of physical harm (some projects may not have any
    appreciable risks of this form), list at least 2 other types of risk in
    software products. Why are they important to consider?
\end{enumerate}

\end{document}