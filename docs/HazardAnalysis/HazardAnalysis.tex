\documentclass{article}

\usepackage{booktabs}
\usepackage{tabularx}
\usepackage{hyperref}

\hypersetup{
    colorlinks=true,       % false: boxed links; true: colored links
    linkcolor=red,          % color of internal links (change box color with linkbordercolor)
    citecolor=green,        % color of links to bibliography
    filecolor=magenta,      % color of file links
    urlcolor=cyan           % color of external links
}

\title{Hazard Analysis\\\progname}

\author{\authname}

\date{}

%% Comments

\usepackage{color}

\newif\ifcomments\commentstrue %displays comments
%\newif\ifcomments\commentsfalse %so that comments do not display

\ifcomments
\newcommand{\authornote}[3]{\textcolor{#1}{[#3 ---#2]}}
\newcommand{\todo}[1]{\textcolor{red}{[TODO: #1]}}
\else
\newcommand{\authornote}[3]{}
\newcommand{\todo}[1]{}
\fi

\newcommand{\wss}[1]{\authornote{magenta}{SS}{#1}} 
\newcommand{\plt}[1]{\authornote{cyan}{TPLT}{#1}} %For explanation of the template
\newcommand{\an}[1]{\authornote{cyan}{Author}{#1}}

%% Common Parts

\newcommand{\progname}{ProgName} % PUT YOUR PROGRAM NAME HERE
\newcommand{\authname}{Team \#, Team Name
\\ Student 1 name
\\ Student 2 name
\\ Student 3 name
\\ Student 4 name} % AUTHOR NAMES                  

\usepackage{hyperref}
    \hypersetup{colorlinks=true, linkcolor=blue, citecolor=blue, filecolor=blue,
                urlcolor=blue, unicode=false}
    \urlstyle{same}
                                


\begin{document}

\maketitle
\thispagestyle{empty}

~\newpage

\pagenumbering{roman}

\begin{table}[hp]
\caption{Revision History} \label{TblRevisionHistory}
\begin{tabularx}{\textwidth}{llX}
\toprule
\textbf{Date} & \textbf{Developer(s)} & \textbf{Change}\\
\midrule
October 2nd 2025 & Dylan Garner & Add Initial Hazards\\
October 9th 2025 & Umar Khan & Added Safety and Security requirements in Hazards analysis\\
October 9th 2025 & Omar El Aref & Added System Boundaries and Components\\
October 9th 2025 & Yousef Shahin & Added Roadmap\\
October 10th 2025 & Aswin Kuganesan & Added Critical Assumptions and Failure Mode and Effect Analysis table\\
October 10th 2025 & Aswin Kuganesan & Added Reflection\\
\bottomrule
\end{tabularx}
\end{table}

~\newpage

\tableofcontents

~\newpage

\pagenumbering{arabic}

\wss{You are free to modify this template.}

\section{Introduction}

\wss{You can include your definition of what a hazard is here.}

\section{Scope and Purpose of Hazard Analysis}

\wss{You should say what \textbf{loss} could be incurred because of the
hazards.}

\section{System Boundaries and Components}

\wss{Dividing the system into components will help you brainstorm the hazards.
You shouldn't do a full design of the components, just get a feel for the major
ones.  For projects that involve hardware, the components will typically include
each individual piece of hardware.  If your software will have a database, or an
important library, these are also potential components.}

\section{Critical Assumptions}

\begin{itemize}
    \item Input images are provided in a usable format for the system and are not corrupted when uploaded
    \item Resolution of fragment images is sufficient for meaningful feature extraction using edges and damage patterns
    \item Users of the system are expected to understand the basic functionality of the system and be knowledgeable in religious studies
    \item External APIs and libraries used (image processing/segmentation) will be supported throughout the entire project lifetime.
    \item All images uploaded and stored in the database are images of ancient fragments that the user has been permitted to use
\end{itemize}

\section{Failure Mode and Effect Analysis}

\begin{table}[htbp]
\centering
\scriptsize
\caption{Failure Mode and Effects Analysis (FMEA) for Manuscript Recreation System}
\begin{tabularx}{\textwidth}{p{2.5cm} X X X X}
\toprule
\textbf{Design Function} & \textbf{Failure Modes} & \textbf{Effects of Failure} & \textbf{Causes of Failure} & \textbf{Recommended Action} \\
\midrule

\multirow{7}{*}{\parbox[t]{2cm}{Manuscript \\ Recreation}}
& Image upload fails or corrupted file is accepted 
& Missing or unusable fragment in display
& Invalid file format, incomplete upload, or corrupted image data 
& Validate file format during upload and provide error messages to users \\

& Misalignment tools produce incorrect rotation/zoom 
& User arranges fragments inaccurately, leading to wrong conclusions 
& Calibration error or algorithmic inaccuracy in transformation logic 
& Allow users to undo and redo previous action \\

& Unauthorized user gains access 
& Data breach or unauthorized edits to manuscripts 
& Weak password, unencrypted session, or bypassed authentication 
& Enforce strong passwords and 2 factor authentication \\

& Data corruption or loss 
& User progress and annotations are lost 
& Database write failure or transaction interruption 
& Perform daily backups, implement transaction rollback and recovery mechanisms \\

& False positives in edge similarity 
& Incorrect fragment matches suggested to user 
& Overfitting or poor training data distribution 
& Require human validation, display probability percentages and match explanations \\

& User loses connection during upload or save 
& Session progress lost 
& Network interruption or unstable client connection 
& Autosave state locally, synchronize data once the connection is restored \\

\bottomrule
\end{tabularx}
\end{table}

\section{Safety and Security Requirements}

\wss{Newly discovered requirements.  These should also be added to the SRS.  (A
rationale design process how and why to fake it.)}

\section{Roadmap}

\wss{Which safety requirements will be implemented as part of the capstone timeline?
Which requirements will be implemented in the future?}

\newpage{}

\section*{Appendix --- Reflection}

\begin{enumerate}
    \item While writing this deliverable, the safety requirements were easy to create because the components were well defined. This allowed for us to determine the most likely hazards to occur based on the
    components and their functions. Then, we were able to create safety requirements to prevent these hazards from occurring. We also were able to determine the critical assumptions easily because many of
    them were assumed during the planning of the project and are assumptions that we have discussed in the past.
    \item During this deliverable, we found that the FMEA table was difficult to complete because we had to think of multiple different ways the system could fail and the steps we would take to prevent these failures. This
    required us to think more critically about the system rather than surface level thinking. Creating the roadmap also required us to consider future work and how we would implement the safety requirements. This required us
    discussing the feasibility of implementing several safety features within the project timeline.
    \item The more serious risks were considered first during project planning. This includes risks such as database corruption and image upload failure. These risks were considered first because they would have the most
    significant impact on the user and the system. Our discussions with our supervisor about requirements led to us thinking about potential risks that we had not considered. This included the risks considered 
    during this deliverable, specifically connection loss and false positives in the algorithm. These risks were considered because although they do not have as big of an impact as the more serious risks, they 
    could still cause the user to experience dissatisfaction with the system. 
    \item Many software applications store user data and rely on this data for their application to personalize the experience for the user. This means that a large amount of user data is stored in databases which need to
    be protected to prevent data breaches. Bad actors may try to gain access to this data for malicious purposes. This is why security is a major concern for software applications. Furthermore, software applications store
    important data that users do not want to lose. Data loss can lead to users losing progress on their work which can have devastating consequences. This is why data integrity needs to be considered when designing a 
    software application. 
\end{enumerate}

\end{document}