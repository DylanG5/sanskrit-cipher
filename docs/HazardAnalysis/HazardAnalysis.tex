\documentclass{article}

\usepackage{booktabs}
\usepackage{tabularx}
\usepackage{hyperref}

\hypersetup{
    colorlinks=true,       % false: boxed links; true: colored links
    linkcolor=red,          % color of internal links (change box color with linkbordercolor)
    citecolor=green,        % color of links to bibliography
    filecolor=magenta,      % color of file links
    urlcolor=cyan           % color of external links
}

\title{Hazard Analysis\\\progname}

\author{\authname}

\date{}

%% Comments

\usepackage{color}

\newif\ifcomments\commentstrue %displays comments
%\newif\ifcomments\commentsfalse %so that comments do not display

\ifcomments
\newcommand{\authornote}[3]{\textcolor{#1}{[#3 ---#2]}}
\newcommand{\todo}[1]{\textcolor{red}{[TODO: #1]}}
\else
\newcommand{\authornote}[3]{}
\newcommand{\todo}[1]{}
\fi

\newcommand{\wss}[1]{\authornote{magenta}{SS}{#1}} 
\newcommand{\plt}[1]{\authornote{cyan}{TPLT}{#1}} %For explanation of the template
\newcommand{\an}[1]{\authornote{cyan}{Author}{#1}}

%% Common Parts

\newcommand{\progname}{ProgName} % PUT YOUR PROGRAM NAME HERE
\newcommand{\authname}{Team \#, Team Name
\\ Student 1 name
\\ Student 2 name
\\ Student 3 name
\\ Student 4 name} % AUTHOR NAMES                  

\usepackage{hyperref}
    \hypersetup{colorlinks=true, linkcolor=blue, citecolor=blue, filecolor=blue,
                urlcolor=blue, unicode=false}
    \urlstyle{same}
                                


\begin{document}

\maketitle
\thispagestyle{empty}

~\newpage

\pagenumbering{roman}

\begin{table}[hp]
\caption{Revision History} \label{TblRevisionHistory}
\begin{tabularx}{\textwidth}{llX}
\toprule
\textbf{Date} & \textbf{Developer(s)} & \textbf{Change}\\
\midrule
October 9, 2025 & Muhammad Umar Khan & Adding Safety and Security Requirements\\
Date2 & Name(s) & Description of changes\\
... & ... & ...\\
\bottomrule
\end{tabularx}
\end{table}

~\newpage

\tableofcontents

~\newpage

\pagenumbering{arabic}

\wss{You are free to modify this template.}

\section{Introduction}

\wss{You can include your definition of what a hazard is here.}

\section{Scope and Purpose of Hazard Analysis}

\wss{You should say what \textbf{loss} could be incurred because of the
hazards.}

\section{System Boundaries and Components}

\wss{Dividing the system into components will help you brainstorm the hazards.
You shouldn't do a full design of the components, just get a feel for the major
ones.  For projects that involve hardware, the components will typically include
each individual piece of hardware.  If your software will have a database, or an
important library, these are also potential components.}

\section{Critical Assumptions}

\wss{These assumptions that are made about the software or system.  You should
minimize the number of assumptions that remove potential hazards.  For instance,
you could assume a part will never fail, but it is generally better to include
this potential failure mode.}

\section{Failure Mode and Effect Analysis}

\wss{Include your FMEA table here. This is the most important part of this document.}
\wss{The safety requirements in the table do not have to have the prefix SR.
The most important thing is to show traceability to your SRS. You might trace to
requirements you have already written, or you might need to add new
requirements.}
\wss{If no safety requirement can be devised, other mitigation strategies can be
entered in the table, including strategies involving providing additional
documentation, and/or test cases.}

\section{Safety and Security Requirements}

The following safety and security requirements are derived from the identified hazards and their recommended mitigations. Each requirement corresponds to one or more failure modes discussed in the previous section. The intent is to ensure that failures in individual services or user actions do not compromise data integrity, security, or scholarly validity.

\begin{itemize}

    \item \textbf{File Integrity and Upload Validation}
    \begin{itemize}
        \item The system shall validate all uploaded files for acceptable format (JPG, PNG) and integrity before ingestion.
        \item The system shall reject corrupted or incomplete files and provide descriptive feedback to the user.
        \item The system shall maintain a checksum for each uploaded image to prevent future corruption or mismatch.
    \end{itemize}

    \item \textbf{User Interaction Safety}
    \begin{itemize}
        \item The system shall include undo/redo functionality for all manual alignment or manipulation tools on the interactive canvas.
        \item The system shall display alignment guides or grid overlays to minimize misplacement of fragments.
        \item The system shall maintain a revision history of all user adjustments to allow recovery from accidental operations.
    \end{itemize}

    \item \textbf{Authentication and Access Control}
    \begin{itemize}
        \item The system shall enforce strong password policies, require secure session tokens, and store credentials in an encrypted manner.
        \item The system shall support optional two-factor authentication (2FA) for privileged or administrative accounts.
        \item The system shall log all authentication events (login, failed attempt, password reset) and perform periodic security audits.
        \item The system shall automatically lock an account after multiple failed authentication attempts within a short time window.
    \end{itemize}

    \item \textbf{Database Integrity and Recovery}
    \begin{itemize}
        \item The system shall perform automated daily backups of all databases (fragment metadata, annotations, user projects).
        \item The system shall implement transaction rollback and atomic commits to prevent partial data writes.
        \item The system shall provide administrators with a verified recovery mechanism for restoring corrupted or lost data.
        \item The system shall maintain referential integrity across related tables to prevent inconsistent state.
    \end{itemize}

    \item \textbf{Service Reliability and Fault Tolerance}
    \begin{itemize}
        \item The system shall detect service timeouts or crashes (e.g., ML inference, orchestration tasks) and automatically retry with exponential backoff.
        \item The system shall provide user feedback when a service is unavailable and allow fallback to manual workflows.
        \item The system shall implement failover handling for long-running processes to avoid complete job loss.
        \item The system shall log all failures and exception traces for later analysis.
    \end{itemize}

    \item \textbf{Model Transparency and Scholarly Control}
    \begin{itemize}
        \item The system shall display confidence levels and similarity metrics for all model-generated suggestions (matching, classification, transcription).
        \item The system shall require explicit human confirmation before accepting any automated fragment match.
        \item The system shall maintain version control for model outputs, ensuring that updated models do not overwrite prior validated results.
    \end{itemize}

    \item \textbf{Network and Connectivity Safety}
    \begin{itemize}
        \item The system shall autosave the current workspace and session data locally every 60 seconds.
        \item The system shall sync unsaved data automatically upon reconnection.
        \item The system shall warn users of unsent uploads or unsynced work before closing the browser or session.
    \end{itemize}

    \item \textbf{Security of Stored and Transmitted Data}
    \begin{itemize}
        \item The system shall use TLS (Transport Layer Security) for all client–server communication and encrypt sensitive data in storage.
        \item The system shall sanitize and validate all user inputs to prevent injection or cross-site scripting (XSS) attacks.
        \item The system shall segregate user data using role-based access to prevent unauthorized access or leakage.
    \end{itemize}

    \item \textbf{Auditability and Traceability}
    \begin{itemize}
        \item The system shall maintain an audit log for all critical operations, including uploads, edits, deletions, and administrative actions.
        \item The system shall associate every change with a user identity and timestamp.
        \item The system shall provide administrators with a secure interface to review audit logs.
    \end{itemize}

\end{itemize}

\section{Roadmap}

\wss{Which safety requirements will be implemented as part of the capstone timeline?
Which requirements will be implemented in the future?}

\newpage{}

\section*{Appendix --- Reflection}

\wss{Not required for CAS 741}

The purpose of reflection questions is to give you a chance to assess your own
learning and that of your group as a whole, and to find ways to improve in the
future. Reflection is an important part of the learning process.  Reflection is
also an essential component of a successful software development process.  

Reflections are most interesting and useful when they're honest, even if the
stories they tell are imperfect. You will be marked based on your depth of
thought and analysis, and not based on the content of the reflections
themselves. Thus, for full marks we encourage you to answer openly and honestly
and to avoid simply writing ``what you think the evaluator wants to hear.''

Please answer the following questions.  Some questions can be answered on the
team level, but where appropriate, each team member should write their own
response:


\begin{enumerate}
    \item What went well while writing this deliverable? 
    \item What pain points did you experience during this deliverable, and how
    did you resolve them?
    \item Which of your listed risks had your team thought of before this
    deliverable, and which did you think of while doing this deliverable? For
    the latter ones (ones you thought of while doing the Hazard Analysis), how
    did they come about?
    \item Other than the risk of physical harm (some projects may not have any
    appreciable risks of this form), list at least 2 other types of risk in
    software products. Why are they important to consider?
\end{enumerate}

\end{document}